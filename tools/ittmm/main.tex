\documentclass[60x84/16,8pt]{ittmm}

\input{definition}

\begin{document}

% Укажите индекс УДК, соответствующий Вашей работе.
% Рукопись должна содержать УДК, который рекомендуется брать из следующего источника: \url{http://www.mathnet.ru/udc.pdf}.
\udc{004.8}

\title{Разработка и исследование модели для генерации субъективно привлекательных лиц людей}

\author[1,2]{М. А. Ким}
\author[1]{М. В. Хачумов}

\address[1]{Кафедра математического моделирования и искусственного интеллекта,\\
  Российский университет дружбы народов,\\
  ул. Миклухо-Маклая, д.6, Москва, Россия, 117198}

\email{\url{1032201664@pfur.ru}}

\begin{abstract}
Когнитивные способности человека позволяют осуществлять восприятие, анализ и внешнюю оценку собеседника.
При этом объяснение субъективной оценки внешности нельзя назвать тривиальной задачей.
Помимо известных общих закономерностей, таких как принципы геометрии лица, социальный и культурный опыт оценивающего,
каждому человеку присуще комплексные индивидуальные факторы, влияющие на субъективное понимание красоты.
Настоящая статья посвящена задаче построения модели машинного обучения, способной в соответствии с обратной
связью от пользователя генерировать субъективно привлекательные лица людей. Предполагается, что такая модель
поможет в изучении процесса формирования человеком оценки привлекательности лиц и выявлении индивидуальных факторов,
влияющих на данный показатель. В работе рассмотрен сервис, предоставляющий пользователю последовательный выбор наиболее привлекательных изображений,
созданных при помощи модели генеративно-состязательной сети (GAN). По окончании заданной длины последовательности изображений
пользователю дается возможность оценить результирующие изображения на предмет привлекательности лиц по заданному показателю качества.
Обучение нейронной сети производилось на датасете CelebA-HQ, который содержит 30000 1024×1024 изображений лиц знаменитостей.
Для увеличения выборки пользователей сервис развернут в виде web-приложения и Telegram бота со свободным доступом, что позволяет
собирать данные о сгенерированных результирующих изображениях. В статье даются выводы о качестве генерации изображений по мнению
пользователей и о возможностях применения полученных результатов.
\end{abstract}

\keywords{изображения лиц, визуализация, генеративные состязательные сети, машинное обучение}

\alttitle{Development and research of a model for generating subjectively attractive human faces}

\altauthor[1,2]{M. A. Kim}
\altauthor[1]{M. V. Khachumov}

\altaddress[1]{Department of Information Technologies\\
Peoples' Friendship University of Russia\\
Miklukho-Maklaya str. 6, Moscow, 117198, Russia}

\begin{altabstract}
A person's cognitive capabilities allow for the perception, analysis, and external evaluation of others.
At the same time, describing the subjective assessment of attractiveness cannot be considered a trivial task.
In addition to well-established general patterns, including the principles of facial geometry and the social and cultural experiences of the evaluator,
every person has complex, individual factors that influence their subjective understanding of attractiveness.
This paper is focused on developing a machine learning model that can generate subjectively attractive human faces based on user feedback.
It is hypothesized that such a model could be used to study the process of how a person forms their assessment of another's attractiveness
and identify individual factors that contribute to this indicator. This study considers a service that offers users a consistent selection
of the most appealing images generated using a generative adversarial network model.At the end of a specified length of sequence of images,
users are given the opportunity to assess the resulting images based on the attractiveness of the faces using a given quality metric.
The neural network has been trained on the CelebA-HQ dataset, which includes 30,000 images of celebrity faces with a resolution of 1024x1024 pixels.
To increase the size of the user sample, the service has been deployed in the form of a web application and Telegram bot with free access.
This allows users to collect data on the generated images.
This article draws conclusions regarding the quality of the image generation based on user feedback and the potential applications of the results obtained.
  
\end{altabstract}

\altkeywords{face images, visualization, generative adversarial networks, machine learning}

\maketitle

\section{Введение}
\label{sec:intro}

Каким образом возможно оценить красоту? Человек способен мгновенно классифицировать
изображение как привлекательное и даже выразить степень красоты заданного объекта в соответствии с определенной метрикой.
Однако, в большинстве случаев, невозможно произвести количественную оценку избранного решения.
Особенно остро данный аспект проявляется при анализе и генерации человеческих лиц.

Трудности исследования данного направления связаны в первую очередь с тем, что
оценка привлекательности одного и того же объекта разными людьми может сильно различаться
в виду, к примеру, возраста, культуры и пола оценивающего
\cite{facial-attractiveness-1, facial-attractiveness-2, facial-attractiveness-3}.
Безусловно, есть и общие закономерности, связывающие определенные черты лица
и среднюю оценку его привлекательности, такие как его симметричность, приблизительно
равная соотношению носа ко лбу и носа к подбородку \cite{facial-attractiveness-math-1}.
Однако, в общем случае считается, что привлекательность -- скорее субъективная, личностная характеристика,
нежели объективный фактор, справедливый для большинства людей.

Создание визуально привлекательных изображений людей до сих пор
является является сложной задачей в виду затруднительности описания
концепции привлекательности. Большинство исследований в данной области
направлены на выявление определенных закономерностей лица в соответствии
с фотографией \cite{facial-attractiveness-math-2, facial-attractiveness-math-3, facial-attractiveness-math-4},
однако, вероятно, подходы такого рода не позволяют объемлюще воспроизвести
факторы, влияющие на субъективную оценку в виду сложности человеческих суждений
относительно привлекательности. Модели, создающие изображения но основе данных
факторов, не способны полностью отразить эстетическое восприятие человека \cite{attr-models-complicated}.
Это связанно с тем, что генерация объектов ограниченна заданными
общими закономерностями и не учитывает субъективные предпочтения человека.

В данной статье представлен альтернативный метод созданию субъективно привлекательных
изображений лиц людей на основе генеративно-состязательной нейронной сети (GAN) \cite{gan}, обученной на датасете CelebA-HQ.
Коррекция в генерации лиц в соответствии с предпочтениями человека производится
по принципу исследования тензора скрытого состояния нейронной сети \cite{latent-space-exploration}. 

\section{Основная часть} 
\label{sec:base-section}

Основная часть работы должна отражать поэтапное подробное решение
поставленной задачи и может содержать несколько разделов. Здесь
проводятся доказательства и решения выдвинутых положений и задач,
рассматриваются методы их решения, приводится наглядный иллюстративный
материал в виде графиков, таблиц, диаграмм и т.~д.

По требованиям организационного комитета конференции объём одной
представляемой рукописи не должен превышать 3-х страниц. Авторы
обязаны предъявлять повышенные требования к изложению и языку
рукописи, а также подготовке иллюстративного и табличного материалов.
Рукопись представляется на русском языке.  Рекомендуется безличная
форма изложения.

При оформлении рекомендуется пользоваться стандартными окружениями
макропакета \LaTeXe.

<<Ссылочный аппарат>> на формулы реализуется с помощью команд
\verb"\label" и \verb"\eqref".\footnote{Нумероваться будут только те
  формулы, на которые ссылка оформлена с помощью этих команд.}  В
качестве примера приведём формулу
\begin{equation}
a^n+b^n=c^n
\label{eq:Fermat's_Last_Theorem}
\end{equation}
и ссылку на неё \eqref{eq:Fermat's_Last_Theorem}.

Рисунки в рукопись вставляются стандартными средствами \LaTeXe.  В
качестве форматов рисунков рекомендуется использовать файлы типа
\texttt{eps} или \texttt{pdf}, изображение должно быть качественным и
векторным. Разрешение растровой графики должно быть не менее 600 dpi
(лучше 1200 dpi). Шрифт на рисунках не должен быть менее 6 пунктов.
Каждый рисунок должен быть подписан, для этого
используется команда \verb"\caption". 
Как пример см. рис.~\ref{fig:logo}.

\begin{figure}
  \centering
  \includegraphics[width=0.2\linewidth]{embl}
  \caption{Эмблема}
  \label{fig:logo}
\end{figure}

Ниже (см. табл.~\ref{tab:sampletable}) представлен вариант таблицы с
заголовком оформленным с помощью \verb"\caption".

\begin{table}
  \centering
  \caption{Пример небольшой таблицы}
  \label{tab:sampletable}
  \begin{tabular}{|c|c|c|c|c|}
    \hline
    Номер & $X$ & $Y$ & $R$ & Цвет\\
    \hline
    1 &     100  &  170 & 30 & красный\\
    2 &     100  &  90      & 60 & жёлтый\\
    3 &     230  &  250     & 50 & синий\\
    4 &     130  &  240 & 60 & зелёный\\
    5 & 300  &      130 & 30 & зелёный\\
    6 &     200  &  150     & 90 & красный\\
    \hline
  \end{tabular}
\end{table}

Для ссылки на источники необходимо использовать команду \verb"\cite".

Литература может формироваться либо с помощью программы \verb"bibtex",
либо с помощью окружения \verb"thebibliography".
При формировании списка литературы просьба использовать стандарт
ГОСТ~Р7.0.5-2008. Примеры цитирования книги~\cite{mathtensor,
  jones-fogelin:tcqd}, раздела в книге~\cite{Muller2006},
статьи~\cite{Arduengo1991, Booth1962},
материалов конференции~\cite{Hope2005}.

На все источники в списке литературы должны быть ссылки.

\section{Заключение}

Заключение является неотъемлемой частью любой работы. 

Оно должно содержать краткие выводы по результатам исследования,
отражающие новизну и практическую значимость работы, предложения по
использованию ее результатов, оценку её эффективности и качества.


\begin{acknowledgments}
  Работа частично поддержана грантом РФФИ
  \textnumero~16-01-20379.\footnote{Этот раздел статьи может
    отсутствовать.  В него рекомендуется добавлять сведения о
    финансировании работы и выражать благодарности персонам.}
\end{acknowledgments}


\begin{thebibliography}{99}

\bibitem{facial-attractiveness-1}
M. R. Cunningham, A. R. Roberts, A. P. Barbee, P. B. Druen and C.-H. Wu.
Their ideas of beauty are on the whole the same as ours. 
--- Vol. 68, no. 2.

\bibitem{facial-attractiveness-2}
J. H. Langlois, J. M. Ritter, L. A. Roggman and L. S. Vaughn.
Facial diversity and infant preferences for attractive faces. 
--- Vol. 27, no. 1.

\bibitem{facial-attractiveness-3}
R. Thornhill and S. W. Gangestad.
Facial attractiveness. 
--- Vol. 3, no. 12. 
--- P. 452-460.

\bibitem{facial-attractiveness-math-1}
K. Schmid, D. Marx and A. Samal.
Computation of a face attractiveness index based on neoclassical canons, symmetry, and golden ratios. 
--- Vol. 41, no. 8.

\bibitem{facial-attractiveness-math-2}
A. C. Little, B. C. Jones and L. M. DeBruine.
Facial attractiveness: Evolutionary based research. 
--- Vol. 366, no. 1571.
--- P. 1638-1659.

\bibitem{facial-attractiveness-math-3}
D. Perrett, K. A. May and S. Yoshikawa.
Facial shape and judgements of female attractiveness.
--- Vol. 368.
--- P. 239-242.

\bibitem{facial-attractiveness-math-4}
J. Shi, A. Samal and D. Marx.
How effective are landmarks and their geometry for face recognition?. 
--- Vol. 102, no. 2.
--- P. 117-133.

\bibitem{attr-models-complicated}
M. Ibáñez-Berganza, A. Amico and V. Loreto.
Subjectivity and complexity of facial attractiveness. 
--- Vol. 9, no. 1.
--- P. 1-12.

\bibitem{gan}
I. Goodfellow et al..
Generative adversarial nets. 
--- P. 2672-2680.

\bibitem{latent-space-exploration}
Huiting Yang, Liangyu Chai, Qiang Wen, Shuang Zhao, Zixun Sun and Shengfeng He.
Discovering interpretable latent space directions of gans beyond binary attributes. 
--- Vol. 33.



% \bibitem{mathtensor}
% L.~Parker, S.~M. Christensen, MathTensor: a system for doing tensor analysis by
%   computer, Addison-Wesley, 1994.

% \bibitem{jones-fogelin:tcqd}
% W.~T. Jones, R.~J. Fogelin, The Twentieth Century to Quine and Derrida, A
%   History of Western Philosophy, Harcourt Brace College Publishers, 1997.

% \bibitem{Muller2006}
% G.~M. Sheldrick, A Short History of SHELXL, International Union of
%   Crystallography and Oxford University Press, 2006.

% \bibitem{Arduengo1991}
% A.~J. Arduengo, III, R.~L. Harlow, M.~Kline, A stable crystalline carbene,
%   J.~Am. Chem. Soc. 113~(1)  361--363.
% \newblock \href {http://dx.doi.org/10.1021/ja00001a054}
%   {\path{doi:10.1021/ja00001a054}}.

% \bibitem{Booth1962}
% G.~Booth, J.~Chatt, The reactions of carbon monoxide and nitric oxide with
%   tertiary phosphine complexes of iron({II}), cobalt({II}), and nickel({II}),
%   J.~Chem. Soc.  2099--2106. \href {http://dx.doi.org/10.1039/JR9620002099}
%   {\path{doi:10.1039/JR9620002099}}.

% \bibitem{Hope2005}
% E.~Hope, J.~Bennett, A.~Stuart, Fluorous zirconium phosphonates: novel
%   inorganic supports for catalysis, in: Pacifichem (International Chemical
%   Congress of Pacific Basin Societies), no. 961, Pacific Basin Chemical
%   Societies.

\end{thebibliography}


% Возможно использовать bibtex.
% \bibliographystyle{ugost2008l}
% \bibliography{main}

\makealttitle

\end{document}
